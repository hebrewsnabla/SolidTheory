%\documentclass[UTF8]{ctexart} % use larger type; default would be 10pt
\documentclass[a4paper]{article}
\usepackage{hw}

%ASSP
\DeclareMathOperator{\tb}{\text{TB}}

\newcommand{\dis}{\displaystyle}
\numberwithin{equation}{section}

\title{固体理论, Homework 06}
\author{王石嵘 20110220098}
\date{\today} % Activate to display a given date or no date (if empty),
         % otherwise the current date is printed 

\begin{document}
% \boldmath
\maketitle

%\tableofcontents

%\newpage

%\setcounter{section}{-1}

\section{}
证明算符$ \e^{i\phi(x)+i\theta(x)} $和$  \e^{i\phi(x')+i\theta(x')} $当$ x\neq x' $时反对易。
\paragraph{Solution:}
Since
\begin{equation}\label{key}
	\e^{\I\alpha \phi_R(x)} \e^{\I\beta \phi_R(y)} = \e^{\I\beta \phi_R(y)}\e^{\I\alpha \phi_R(x)} \e^{-\I\alpha\beta\pi\;\mathrm{sgn}(x-y)}
\end{equation}
we have
\begin{align}\label{key}
\{\e^{\I\phi(x)+\I\theta(x)}, \e^{\I\phi(x')+\I\theta(x')}\} &= \e^{\I\phi(x)+\I\theta(x)} \e^{i\phi(x')+\I\theta(x')} + \e^{\I\phi(x')+\I\theta(x')} \e^{\I\phi(x)+\I\theta(x)} \notag\\
&= \e^{\I\phi(x)+\I\theta(x)} \e^{\I\phi(x')+\I\theta(x')} + \e^{\I\phi(x)+\I\theta(x)} \e^{\I\phi(x')+\I\theta(x')} \e^{-\I \pi\; \mathrm{sgn}(x'-x)} \notag\\
&= \e^{\I\phi(x)+\I\theta(x)} \e^{\I\phi(x')+\I\theta(x')} [1 +  \e^{-\I \pi\; \mathrm{sgn}(x'-x)} ]
\end{align}
when $ x \neq x' $,
\begin{align}\label{key}
\{\e^{\I\phi(x)+\I\theta(x)}, \e^{\I\phi(x')+\I\theta(x')}\} 
	&= \e^{\I\phi(x)+\I\theta(x)} \e^{\I\phi(x')+\I\theta(x')} [1 +  \e^{-\I \pi(\pm 1)} ] \notag\\
	&= \e^{\I\phi(x)+\I\theta(x)} \e^{\I\phi(x')+\I\theta(x')} [1 +  (-1)] \notag\\
	&= 0
\end{align}

\section{Umklapp过程:}
考虑一个相互作用,将两个左侧费米点的费米子散射到右侧费米点,或者反过来。a)这样的相互作用对$ k_F $有什么要求?b)写下这个相互作用对应的哈密顿量的二次量子化形式。c)用玻色化方法将该相互作用项用玻色化之后的标量场表示出来。
\paragraph{Solution:}
\subparagraph{a)}
$ 	k_F = \pi/2 $
\subparagraph{b)}
\begin{equation}\label{key}
	H_u = v \int \dd x [\Psi_R^\dagger \Psi_R^\dagger \Psi_L\Psi_L + \Psi_L\Psi_L \Psi_R^\dagger \Psi_R^\dagger]
\end{equation}

\subparagraph{c)}

\end{document}