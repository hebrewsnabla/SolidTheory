%\documentclass[UTF8]{ctexart} % use larger type; default would be 10pt
\documentclass[a4paper]{article}
\usepackage{xeCJK}
%\usepackage{ctex}
%\usepackage{luatexja-fontspec}
%\setmainjfont{FandolSong}
%\usepackage[utf8]{inputenc} % set input encoding (not needed with XeLaTeX)

%%% Examples of Article customizations
% These packages are optional, depending whether you want the features they provide.
% See the LaTeX Companion or other references for full information.

%%% PAGE DIMENSIONS
\usepackage{geometry} % to change the page dimensions
\geometry{a4paper} % or letterpaper (US) or a5paper or....
\geometry{margin=1.2in} % for example, change the margins to 2 inches all round
% \geometry{landscape} % set up the page for landscape
%   read geometry.pdf for detailed page layout information

\usepackage{graphicx} % support the \includegraphics command and options

% \usepackage[parfill]{parskip} % Activate to begin paragraphs with an empty line rather than an indent

%%% PACKAGES
\usepackage{booktabs} % for much better looking tables
\usepackage{array} % for better arrays (eg matrices) in maths
\usepackage{paralist} % very flexible & customisable lists (eg. enumerate/itemize, etc.)
\usepackage{verbatim} % adds environment for commenting out blocks of text & for better verbatim
\usepackage{subfig} % make it possible to include more than one captioned figure/table in a single float
% These packages are all incorporated in the memoir class to one degree or another...

%%% HEADERS & FOOTERS
\usepackage{fancyhdr} % This should be set AFTER setting up the page geometry
\pagestyle{fancy} % options: empty , plain , fancy
\renewcommand{\headrulewidth}{0pt} % customise the layout...
\lhead{}\chead{}\rhead{}
\lfoot{}\cfoot{\thepage}\rfoot{}

%%% SECTION TITLE APPEARANCE
\usepackage{sectsty}
%\allsectionsfont{\sffamily\mdseries\upshape} % (See the fntguide.pdf for font help)
% (This matches ConTeXt defaults)

%%% ToC (table of contents) APPEARANCE
\usepackage[nottoc,notlof,notlot]{tocbibind} % Put the bibliography in the ToC
\usepackage[titles,subfigure]{tocloft} % Alter the style of the Table of Contents
%\renewcommand{\cftsecfont}{\rmfamily\mdseries\upshape}
%\renewcommand{\cftsecpagefont}{\rmfamily\mdseries\upshape} % No bold!

%%% END Article customizations

%%% The "real" document content comes below...

\setlength{\parindent}{0pt}
\usepackage{physics}
\usepackage{amsmath}
%\usepackage{symbols}
\usepackage{amsfonts}
\usepackage{bm}
%\usepackage{eucal}
\usepackage{mathrsfs}
\usepackage{amssymb}
\usepackage{float}
\usepackage{multicol}
\usepackage{abstract}
\usepackage{empheq}
\usepackage{extarrows}
\usepackage{textcomp}
\usepackage{mhchem}
\usepackage{braket}
\usepackage{siunitx}
\usepackage[utf8]{inputenc}
\usepackage{tikz-feynman}
\usepackage{feynmp}
\usepackage{fontspec}
%\sffamily
%\setmainfont{CMU Serif}
%\setsansfont{CMU Sans Serif}
%\setmonofont{CMU Typewriter Text}




\DeclareMathOperator{\p}{\prime}
\DeclareMathOperator{\ti}{\times}

\DeclareMathOperator{\e}{\mathrm{e}}
\DeclareMathOperator{\I}{\mathrm{i}}
\DeclareMathOperator{\Arg}{\mathrm{Arg}}
\newcommand{\NA}{N_\mathrm{A}}
\newcommand{\kB}{k_\mathrm{B}}

\DeclareMathOperator{\ra}{\rightarrow}
\DeclareMathOperator{\llra}{\longleftrightarrow}
\DeclareMathOperator{\lra}{\longrightarrow}
\DeclareMathOperator{\dlra}{\;\Leftrightarrow\;}
\DeclareMathOperator{\dra}{\;\Rightarrow\;}

%%%%%%%%%%%% QUANTUM MECHANICS %%%%%%%%%%%%%%%%%%%%%%%%
\newcommand{\bkk}[1]{\Braket{#1|#1}}
\newcommand{\bk}[2]{\Braket{#1|#2}}
\newcommand{\bkev}[2]{\Braket{#2|#1|#2}}

\DeclareMathOperator{\na}{\bm{\nabla}}
\DeclareMathOperator{\nna}{\nabla^2}
\DeclareMathOperator{\drrr}{\dd[3]\vb{r}}

\DeclareMathOperator{\psis}{\psi^\ast}
\DeclareMathOperator{\Psis}{\Psi^\ast}
\DeclareMathOperator{\hi}{\hat{\vb{i}}}
\DeclareMathOperator{\hj}{\hat{\vb{j}}}
\DeclareMathOperator{\hk}{\hat{\vb{k}}}
\DeclareMathOperator{\hr}{\hat{\vb{r}}}
\DeclareMathOperator{\hT}{\hat{\vb{T}}}
\DeclareMathOperator{\hH}{\hat{H}}

\DeclareMathOperator{\hL}{\hat{\vb{L}}}
\DeclareMathOperator{\hp}{\hat{\vb{p}}}
\DeclareMathOperator{\hx}{\hat{\vb{x}}}
\DeclareMathOperator{\ha}{\hat{\vb{a}}}
\DeclareMathOperator{\hs}{\hat{\vb{s}}}
\DeclareMathOperator{\hS}{\hat{\vb{S}}}
\DeclareMathOperator{\hSigma}{\hat{\bm\Sigma}}
\DeclareMathOperator{\hJ}{\hat{\vb{J}}}

\DeclareMathOperator{\Tdv}{-\dfrac{\hbar^2}{2m}\dv[2]{x}}
\DeclareMathOperator{\Tna}{-\dfrac{\hbar^2}{2m}\nabla^2}

%\DeclareMathOperator{\s}{\sum_{n=1}^{\infty}}
\DeclareMathOperator{\intinf}{\int_0^\infty}
\DeclareMathOperator{\intdinf}{\int_{-\infty}^\infty}
%\DeclareMathOperator{\suminf}{\sum_{n=0}^\infty}
\DeclareMathOperator{\sumnzinf}{\sum_{n=0}^\infty}
\DeclareMathOperator{\sumnoinf}{\sum_{n=1}^\infty}
\DeclareMathOperator{\sumndinf}{\sum_{n=-\infty}^\infty}
\DeclareMathOperator{\sumizinf}{\sum_{i=0}^\infty}

%%%%%%%%%%%%%%%%% PARTICLE PHYSICS %%%%%%%%%%%%%%%%
\DeclareMathOperator{\hh}{\hat{h}}               % helicity
\DeclareMathOperator{\hP}{\hat{\vb{P}}}          % Parity
\DeclareMathOperator{\hU}{\hat{U}}
\DeclareMathOperator{\hG}{\hat{G}}

\DeclareMathOperator{\GeV}{\si{GeV}}
\DeclareMathOperator{\LI}{\mathscr{L}.I.}
%\DeclareMathOperator{\g5}{\gamma^5}
\DeclareMathOperator{\gmuu}{\gamma^\mu}
\DeclareMathOperator{\gmud}{\gamma_\mu}
\DeclareMathOperator{\gnuu}{\gamma^\nu}
\DeclareMathOperator{\gnud}{\gamma_\nu}

\renewcommand{\u}{\mathrm{u}}
\renewcommand{\d}{\mathrm{d}}
\DeclareMathOperator{\s}{\mathrm{s}}

\DeclareMathOperator{\q}{\mathrm{q}}
\DeclareMathOperator{\bq}{\bar{\mathrm{q}}}

\DeclareMathOperator{\g}{\mathrm{g}}
\DeclareMathOperator{\W}{\mathrm{W}}
\DeclareMathOperator{\Z}{\mathrm{Z}}

%%% Feynman Diagram
\newcommand{\pa}{particle}
\newcommand{\mo}{momentum}
\newcommand{\el}{edge label}

%%% MQC
\DeclareMathOperator{\sH}{\mathscr{H}}
\DeclareMathOperator{\sA}{\mathscr{A}}
\newcommand{\iden}{{\large \bm{1}}}
\newcommand{\qed}{$ \Square $}
\newcommand{\tPhi}{\tilde{\Phi} }
\newcommand{\hsP}{\hat{\mathscr{P}}}
\newcommand{\hsS}{\hat{\mathscr{S}}}
\DeclareMathOperator{\core}{\mathrm{core}}
\DeclareMathOperator{\corr}{\mathrm{corr}}
\DeclareMathOperator{\ext}{\mathrm{ext}}

%ASSP
\DeclareMathOperator{\tb}{\text{TB}}

\newcommand{\dis}{\displaystyle}
\numberwithin{equation}{section}

\title{固体理论, Homework 03}
\author{王石嵘 20110220098}
\date{\today} % Activate to display a given date or no date (if empty),
         % otherwise the current date is printed 

\begin{document}
% \boldmath
\maketitle

%\tableofcontents

%\newpage

%\setcounter{section}{-1}


\section{教材习题7.2}
Calculate explicitly the $ \varepsilon_{n\sigma} $s in the Hartree–Fock approximation to the Anderson model.
\paragraph{Solution:}
From textbook, we have
\begin{align}
\varepsilon_{n\sigma} \Braket{\vb{k}\sigma | n\sigma} &= \varepsilon_{\vb{k}} \Braket{\vb{k}\sigma | n\sigma} + V_{\vb{k}d} \Braket{d\sigma | n\sigma} \label{1}\\
\varepsilon_{n\sigma} \Braket{d\sigma | n\sigma} &= E_{d\sigma} \Braket{d\sigma | n\sigma} + \sum_{\vb{k}}V_{\vb{k}d} \Braket{\vb{k}\sigma | n\sigma} \label{2}
\end{align}
with \eqref{1}, we get
\begin{equation}\label{key}
 \Braket{\vb{k}\sigma | n\sigma} =  \dfrac{V_{\vb{k}d} }{\varepsilon_{n\sigma} - \varepsilon_{\vb{k}} }\Braket{d\sigma | n\sigma}
\end{equation}
plug it into \eqref{2},
\begin{equation}\label{key}
(\varepsilon_{n\sigma} - E_{d\sigma} ) \Braket{d\sigma | n\sigma} =  \sum_{\vb{k}} \dfrac{V_{\vb{k}d}^2 }{\varepsilon_{n\sigma} - \varepsilon_{\vb{k}} }\Braket{d\sigma | n\sigma}
\end{equation}
thus
\begin{equation}\label{key}
\varepsilon_{n\sigma} = E_{d\sigma} + \dfrac{V_{\vb{k}d}^2 }{\varepsilon_{n\sigma} - \varepsilon_{\vb{k}} }
\end{equation}

\section{教材习题7.4}
Show that when $ \varepsilon_d = U/2$ the impurity terms in the Anderson model,
$ \sum_\sigma (\varepsilon_d n_{d\sigma} + U n_{d\uparrow} n_{d\downarrow}) $
, are invariant under the transformation $ a_d^\dagger \leftrightarrow a_d $.
\paragraph{Solution:}
\begin{align}
\sum_\sigma (\varepsilon_d n_{d\sigma} + U n_{d\uparrow} n_{d\downarrow}) &= \varepsilon_d (a_{d\uparrow}^\dagger a_{d\uparrow} + a_{d\downarrow}^\dagger a_{d\downarrow}) + Ua_{d\uparrow}^\dagger a_{d\uparrow}a_{d\downarrow}^\dagger a_{d\downarrow} \notag\\
&= -\dfrac{U}{2} (a_{d\uparrow}^\dagger a_{d\uparrow} + a_{d\downarrow}^\dagger a_{d\downarrow}) + Ua_{d\uparrow}^\dagger a_{d\uparrow}a_{d\downarrow}^\dagger a_{d\downarrow} \notag\\
&= U\qty( -\dfrac{1}{2}a_{d\uparrow}^\dagger a_{d\uparrow} - \dfrac{1}{2}a_{d\downarrow}^\dagger a_{d\downarrow}  + a_{d\uparrow}^\dagger a_{d\uparrow}a_{d\downarrow}^\dagger a_{d\downarrow})
\end{align}
after the transformation $ a_d^\dagger \leftrightarrow a_d $,
\begin{align}
\sum_\sigma (\varepsilon_d n_{d\sigma} + U n_{d\uparrow} n_{d\downarrow}) &= -\dfrac{U}{2} ( a_{d\uparrow}a_{d\uparrow}^\dagger +  a_{d\downarrow}a_{d\downarrow}^\dagger) + U a_{d\uparrow}a_{d\uparrow}^\dagger a_{d\downarrow}a_{d\downarrow}^\dagger \notag\\
&= -\dfrac{U}{2} (1 - a_{d\uparrow}^\dagger a_{d\uparrow} + 1 - a_{d\downarrow}^\dagger a_{d\downarrow}) + U(1 - a_{d\uparrow}^\dagger a_{d\uparrow})(1 - a_{d\downarrow}^\dagger a_{d\downarrow}) \notag\\
&= U\qty(-1 +\dfrac{1}{2}a_{d\uparrow}^\dagger a_{d\uparrow} + \dfrac{1}{2}a_{d\downarrow}^\dagger a_{d\downarrow} + 1 - a_{d\uparrow}^\dagger a_{d\uparrow} - a_{d\downarrow}^\dagger a_{d\downarrow} + a_{d\uparrow}^\dagger a_{d\uparrow}a_{d\downarrow}^\dagger a_{d\downarrow}) \notag\\
&= U\qty( -\dfrac{1}{2}a_{d\uparrow}^\dagger a_{d\uparrow} - \dfrac{1}{2}a_{d\downarrow}^\dagger a_{d\downarrow}  + a_{d\uparrow}^\dagger a_{d\uparrow}a_{d\downarrow}^\dagger a_{d\downarrow}) 
\end{align}
$ \therefore $ the impurity terms are invariant under the transformation $ a_d^\dagger \leftrightarrow a_d $ when $ \varepsilon_d = U/2$ 
\end{document}