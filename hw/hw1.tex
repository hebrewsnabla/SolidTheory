%\documentclass[UTF8]{ctexart} % use larger type; default would be 10pt
\documentclass[a4paper]{article}
\usepackage{xeCJK}
%\usepackage{ctex}
%\usepackage{luatexja-fontspec}
%\setmainjfont{FandolSong}
%\usepackage[utf8]{inputenc} % set input encoding (not needed with XeLaTeX)

%%% Examples of Article customizations
% These packages are optional, depending whether you want the features they provide.
% See the LaTeX Companion or other references for full information.

%%% PAGE DIMENSIONS
\usepackage{geometry} % to change the page dimensions
\geometry{a4paper} % or letterpaper (US) or a5paper or....
% \geometry{margin=2in} % for example, change the margins to 2 inches all round
% \geometry{landscape} % set up the page for landscape
%   read geometry.pdf for detailed page layout information

\usepackage{graphicx} % support the \includegraphics command and options

% \usepackage[parfill]{parskip} % Activate to begin paragraphs with an empty line rather than an indent

%%% PACKAGES
\usepackage{booktabs} % for much better looking tables
\usepackage{array} % for better arrays (eg matrices) in maths
\usepackage{paralist} % very flexible & customisable lists (eg. enumerate/itemize, etc.)
\usepackage{verbatim} % adds environment for commenting out blocks of text & for better verbatim
\usepackage{subfig} % make it possible to include more than one captioned figure/table in a single float
% These packages are all incorporated in the memoir class to one degree or another...

%%% HEADERS & FOOTERS
\usepackage{fancyhdr} % This should be set AFTER setting up the page geometry
\pagestyle{fancy} % options: empty , plain , fancy
\renewcommand{\headrulewidth}{0pt} % customise the layout...
\lhead{}\chead{}\rhead{}
\lfoot{}\cfoot{\thepage}\rfoot{}

%%% SECTION TITLE APPEARANCE
\usepackage{sectsty}
%\allsectionsfont{\sffamily\mdseries\upshape} % (See the fntguide.pdf for font help)
% (This matches ConTeXt defaults)

%%% ToC (table of contents) APPEARANCE
\usepackage[nottoc,notlof,notlot]{tocbibind} % Put the bibliography in the ToC
\usepackage[titles,subfigure]{tocloft} % Alter the style of the Table of Contents
%\renewcommand{\cftsecfont}{\rmfamily\mdseries\upshape}
%\renewcommand{\cftsecpagefont}{\rmfamily\mdseries\upshape} % No bold!

%%% END Article customizations

%%% The "real" document content comes below...

\setlength{\parindent}{0pt}
\usepackage{physics}
\usepackage{amsmath}
%\usepackage{symbols}
\usepackage{amsfonts}
\usepackage{bm}
%\usepackage{eucal}
\usepackage{mathrsfs}
\usepackage{amssymb}
\usepackage{float}
\usepackage{multicol}
\usepackage{abstract}
\usepackage{empheq}
\usepackage{extarrows}
\usepackage{textcomp}
\usepackage{mhchem}
\usepackage{braket}
\usepackage{siunitx}
\usepackage[utf8]{inputenc}
\usepackage{tikz-feynman}
\usepackage{feynmp}
\usepackage{fontspec}
%\sffamily
%\setmainfont{CMU Serif}
%\setsansfont{CMU Sans Serif}
%\setmonofont{CMU Typewriter Text}




\DeclareMathOperator{\p}{\prime}
\DeclareMathOperator{\ti}{\times}

\DeclareMathOperator{\e}{\mathrm{e}}
\DeclareMathOperator{\I}{\mathrm{i}}
\DeclareMathOperator{\Arg}{\mathrm{Arg}}
\newcommand{\NA}{N_\mathrm{A}}
\newcommand{\kB}{k_\mathrm{B}}

\DeclareMathOperator{\ra}{\rightarrow}
\DeclareMathOperator{\llra}{\longleftrightarrow}
\DeclareMathOperator{\lra}{\longrightarrow}
\DeclareMathOperator{\dlra}{\;\Leftrightarrow\;}
\DeclareMathOperator{\dra}{\;\Rightarrow\;}

%%%%%%%%%%%% QUANTUM MECHANICS %%%%%%%%%%%%%%%%%%%%%%%%
\newcommand{\bkk}[1]{\Braket{#1|#1}}
\newcommand{\bk}[2]{\Braket{#1|#2}}
\newcommand{\bkev}[2]{\Braket{#2|#1|#2}}

\DeclareMathOperator{\na}{\bm{\nabla}}
\DeclareMathOperator{\nna}{\nabla^2}
\DeclareMathOperator{\drrr}{\dd[3]\vb{r}}

\DeclareMathOperator{\psis}{\psi^\ast}
\DeclareMathOperator{\Psis}{\Psi^\ast}
\DeclareMathOperator{\hi}{\hat{\vb{i}}}
\DeclareMathOperator{\hj}{\hat{\vb{j}}}
\DeclareMathOperator{\hk}{\hat{\vb{k}}}
\DeclareMathOperator{\hr}{\hat{\vb{r}}}
\DeclareMathOperator{\hT}{\hat{\vb{T}}}
\DeclareMathOperator{\hH}{\hat{H}}

\DeclareMathOperator{\hL}{\hat{\vb{L}}}
\DeclareMathOperator{\hp}{\hat{\vb{p}}}
\DeclareMathOperator{\hx}{\hat{\vb{x}}}
\DeclareMathOperator{\ha}{\hat{\vb{a}}}
\DeclareMathOperator{\hs}{\hat{\vb{s}}}
\DeclareMathOperator{\hS}{\hat{\vb{S}}}
\DeclareMathOperator{\hSigma}{\hat{\bm\Sigma}}
\DeclareMathOperator{\hJ}{\hat{\vb{J}}}

\DeclareMathOperator{\Tdv}{-\dfrac{\hbar^2}{2m}\dv[2]{x}}
\DeclareMathOperator{\Tna}{-\dfrac{\hbar^2}{2m}\nabla^2}

%\DeclareMathOperator{\s}{\sum_{n=1}^{\infty}}
\DeclareMathOperator{\intinf}{\int_0^\infty}
\DeclareMathOperator{\intdinf}{\int_{-\infty}^\infty}
%\DeclareMathOperator{\suminf}{\sum_{n=0}^\infty}
\DeclareMathOperator{\sumnzinf}{\sum_{n=0}^\infty}
\DeclareMathOperator{\sumnoinf}{\sum_{n=1}^\infty}
\DeclareMathOperator{\sumndinf}{\sum_{n=-\infty}^\infty}
\DeclareMathOperator{\sumizinf}{\sum_{i=0}^\infty}

%%%%%%%%%%%%%%%%% PARTICLE PHYSICS %%%%%%%%%%%%%%%%
\DeclareMathOperator{\hh}{\hat{h}}               % helicity
\DeclareMathOperator{\hP}{\hat{\vb{P}}}          % Parity
\DeclareMathOperator{\hU}{\hat{U}}
\DeclareMathOperator{\hG}{\hat{G}}

\DeclareMathOperator{\GeV}{\si{GeV}}
\DeclareMathOperator{\LI}{\mathscr{L}.I.}
%\DeclareMathOperator{\g5}{\gamma^5}
\DeclareMathOperator{\gmuu}{\gamma^\mu}
\DeclareMathOperator{\gmud}{\gamma_\mu}
\DeclareMathOperator{\gnuu}{\gamma^\nu}
\DeclareMathOperator{\gnud}{\gamma_\nu}

\renewcommand{\u}{\mathrm{u}}
\renewcommand{\d}{\mathrm{d}}
\DeclareMathOperator{\s}{\mathrm{s}}

\DeclareMathOperator{\q}{\mathrm{q}}
\DeclareMathOperator{\bq}{\bar{\mathrm{q}}}

\DeclareMathOperator{\g}{\mathrm{g}}
\DeclareMathOperator{\W}{\mathrm{W}}
\DeclareMathOperator{\Z}{\mathrm{Z}}

%%% Feynman Diagram
\newcommand{\pa}{particle}
\newcommand{\mo}{momentum}
\newcommand{\el}{edge label}

%%% MQC
\DeclareMathOperator{\sH}{\mathscr{H}}
\DeclareMathOperator{\sA}{\mathscr{A}}
\newcommand{\iden}{{\large \bm{1}}}
\newcommand{\qed}{$ \Square $}
\newcommand{\tPhi}{\tilde{\Phi} }
\newcommand{\hsP}{\hat{\mathscr{P}}}
\newcommand{\hsS}{\hat{\mathscr{S}}}
\DeclareMathOperator{\core}{\mathrm{core}}
\DeclareMathOperator{\corr}{\mathrm{corr}}
\DeclareMathOperator{\ext}{\mathrm{ext}}

\newcommand{\dis}{\displaystyle}
\numberwithin{equation}{section}

\title{固体理论, Homework 01}
\author{王石嵘 20110220098}
\date{\today} % Activate to display a given date or no date (if empty),
         % otherwise the current date is printed 

\begin{document}
% \boldmath
\maketitle

%\tableofcontents

%\newpage

%\setcounter{section}{-1}


\section{简谐振子与二次量子化}
对于一个简谐振子,
\begin{equation}\label{key}
	\hat H = \frac{\hat p^2}{2m}+\frac12m\omega^2\hat x^2
\end{equation}
构造产生及湮灭算符:
\begin{equation}\label{key}
	\hat a = \lambda \hat x + i\mu \hat p,
	\quad \hat a^\dagger = \lambda \hat x - i\mu \hat p
\end{equation}
选择系数$ \lambda $和$\mu $,使得它们满足玻色子产生湮灭算符的对易关系,并且$ H $可以表示成
\begin{equation}\label{key}
	\hat H = \omega\left(\hat a^\dagger \hat a + \frac{1}{2}\right)
\end{equation}
利用产生湮灭算符求解该简谐振子的能谱和能量本征态。
\paragraph{Solution:}
\begin{align}\label{key}
	 [\hat a, \hat a^\dagger] &= (\lambda \hat x + i\mu \hat p)(\lambda \hat x - i\mu \hat p) - (\lambda \hat x - i\mu \hat p)(\lambda \hat x + i\mu \hat p) \notag\\
	 &= \lambda^2 \hat x^2 + i\lambda\mu[\hat p , \hat x] + \mu^2 \hat p^2 - (\lambda^2 \hat x^2 - i\lambda\mu[\hat p , \hat x] + \mu^2 \hat p^2 ) \notag\\
	 &= 2i\lambda\mu[\hat p , \hat x] \notag\\
	 &= 2i\lambda\mu \cdot (-i) \notag\\
	 &= 2\lambda\mu
\end{align}
Since $ [\hat a, \hat a^\dagger] = 1$, we have
\begin{equation}\label{key}
	\mu = \dfrac{1}{2\lambda}
\end{equation}
thus
\begin{align}
	\omega \left(\hat a^\dagger \hat a + \frac{1}{2}\right) &= \omega \qty(\lambda^2 \hat x^2 + \lambda\mu + \mu^2 \hat p^2 + \frac{1}{2}) \notag\\
	&= \omega \qty(\lambda^2 \hat x^2 -\dfrac{1}{2} + \dfrac{1}{4\lambda^2} \hat p^2 + \frac{1}{2}) \notag\\
	&= \omega \qty(\lambda^2 \hat x^2  + \dfrac{1}{4\lambda^2} \hat p^2)
\end{align}
Since $ \omega \left(\hat a^\dagger \hat a + \frac{1}{2}\right) = \hat H $, we get
\begin{equation}\label{key}
	\lambda^2 = \dfrac{1}{2}m\omega
\end{equation}
\begin{equation}\label{key}
	\lambda = \sqrt{\dfrac{1}{2}m\omega} , \quad \mu = \dfrac{1}{\sqrt{2m\omega} }
\end{equation}
i.e.
\begin{equation}\label{key}
	\boxed{
	\hat a = \dfrac{1}{\sqrt{2}}\qty(\sqrt{m\omega}\hat x + i\dfrac{\hat p}{\sqrt{m\omega}}) \quad 
	\hat a^\dagger = \dfrac{1}{\sqrt{2}}\qty(\sqrt{m\omega}\hat x - i\dfrac{\hat p}{\sqrt{m\omega}})
}
\end{equation}


Since 
\begin{equation}\label{key}
	\hat H = \omega \qty(\hat n + \dfrac{1}{2})
\end{equation}
$ \hat H $ commutes with $ \hat n $, thus shares the same eigenstates with $ \hat n $.\\ 
Since 
\begin{equation}\label{key}
	\hat n \ket{n} = n \ket{n}
\end{equation}
we have
\begin{equation}\label{key}
	\hat H \ket{n} = \omega \qty(\hat n + \dfrac{1}{2}) \ket{n} = \omega \qty(n + \dfrac{1}{2}) \ket{n}
\end{equation}
i.e.
\begin{equation}\label{key}
	\boxed{E_n = \omega \qty(n + \dfrac{1}{2}) }
\end{equation}
\begin{align}
	\ket{n} = \dfrac{1}{\sqrt{n!}} (\hat a^\dagger)^n \ket{0}
\end{align}
The ground state could be obtained by
\begin{equation}\label{key}
	\Braket{x | a | 0} = 0
\end{equation}
thus
\begin{align}
	\dfrac{1}{\sqrt{2}} \Bra{x}\qty(\sqrt{m\omega} x + \dfrac{\partial_x}{\sqrt{m\omega}}) \Ket{ 0} = 0
\end{align}
\begin{equation}\label{key}
	\dfrac{\sqrt{m\omega}}{\sqrt{2}} \qty( x + \dfrac{\partial_x}{m\omega}) \Braket{x| 0} = 0
\end{equation}
$ \therefore $
\begin{equation}\label{key}
	\Braket{x|0} = \qty(\dfrac{m\omega}{\pi})^{1/4} \e^{-\frac{1}{2}m\omega x^2}
\end{equation}
thus
\begin{align}
	\Braket{x|n} &= \dfrac{1}{\sqrt{n!}} (\hat a^\dagger)^n \ket{0} \notag\\
	&= \dfrac{1}{\sqrt{n!}} \qty[\sqrt{\dfrac{m\omega}{2}} \qty(x + \dfrac{\partial_x}{m\omega})]^n \Braket{x|0}
\end{align}
\begin{equation}\label{key}
	\boxed{	\Braket{x|n} = \dfrac{1}{\sqrt{2^n n!}} H_n(\sqrt{m\omega}x) \Braket{x|0}
	}
\end{equation}
\section{}
考虑一个由三个电子组成的体系,三个电子分别占据$ \phi_1, \phi_2 $和$ \phi_3 $三个轨道。
\begin{enumerate}
	\item 写下一次量子化的费米子波函数$ \Psi(r_1,r_2,r_3) $并计算体系的能量(包括动能,势能及电子库伦相互作用能)。
	\item 写下二次量子化的费米子波函数并计算体系的能量,并与上一问中结果进行比较。
\end{enumerate}
\paragraph{Solution:}~\\
\paragraph{1.}
\begin{equation}\label{key}
	\Psi(r_1,r_2,r_3) = \dfrac{1}{\sqrt{6}}\mqty|\phi_1(r_1,\sigma_1) & \phi_1(r_2,\sigma_2) & \phi_1(r_3,\sigma_3) \\ \phi_2(r_1,\sigma_1) & \phi_2(r_2,\sigma_2) & \phi_2(r_3,\sigma_3) \\ \phi_3(r_1,\sigma_1) & \phi_3(r_2,\sigma_2) & \phi_3(r_3,\sigma_3) |
\end{equation}
\begin{equation}\label{key}
	E = \Braket{\Psi | \hat H | \Psi} = \sum_{i=1}^3 \Braket{\phi_{i,\sigma_i} | \hat h(i) | \phi_{i,\sigma_i}} + \Braket{12|12} - \Braket{12|21} + \Braket{13|13} - \Braket{13|31} + \Braket{23|23} - \Braket{23|32}
\end{equation}
where
\begin{equation}\label{key}
	\hat h(i) = -\dfrac{1}{2}\nabla^2 + V(i)
\end{equation}
\begin{equation}\label{key}
	\Braket{ij|kl} = \Braket{\phi_i(r_1,\sigma_1)\phi_j(r_2,\sigma_2) | \dfrac{1}{r_{12}} | \phi_k(r_1,\sigma_1) \phi_l(r_2,\sigma_2) }
\end{equation}

\paragraph{2.}
\begin{equation}\label{key}
	\Psi = \hat a_{1,\sigma_1}^\dagger \hat a_{2,\sigma_2}^\dagger \hat a_{3,\sigma_3}^\dagger \ket{0}
\end{equation}
\begin{equation}\label{key}
\hat H = \sum_{i,j=1}^3  \hat a_{i, \sigma_i}^\dagger \Braket{\phi_{i,\sigma_i} | \hat h |\phi_{j,\sigma_j}} \hat a_{j,\sigma_j} 
	+ \dfrac{1}{2}\sum_{i,j,k,l=1}^3 \hat a_{i, \sigma_i}^\dagger \hat a_{j, \sigma_j}^\dagger \Braket{\phi_{i, \sigma_i} \phi_{j, \sigma_j} | \hat g | \phi_{k,\sigma_k} \phi_{l,\sigma_l}} \hat a_{k,\sigma_k} \hat a_{l,\sigma_l}
\end{equation}
Noticing that $ {i,j} = {k,l} $
\begin{equation}\label{key}
	E = \Braket{\Psi | \hat H | \Psi} = \sum_{i=1}^3 \Braket{\phi_{i,\sigma_i} | \hat h(i) | \phi_{i,\sigma_i}} + \Braket{12|12} - \Braket{12|21} + \Braket{13|13} - \Braket{13|31} + \Braket{23|23} - \Braket{23|32}
\end{equation}

\end{document}