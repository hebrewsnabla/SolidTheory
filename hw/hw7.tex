%\documentclass[UTF8]{ctexart} % use larger type; default would be 10pt
\documentclass[a4paper]{article}
\usepackage{hw}

%ASSP
\DeclareMathOperator{\tb}{\text{TB}}

\newcommand{\dis}{\displaystyle}
\numberwithin{equation}{section}

\title{固体理论, Homework 07}
\author{王石嵘 20110220098}
\date{\today} % Activate to display a given date or no date (if empty),
         % otherwise the current date is printed 

\begin{document}
% \boldmath
\maketitle

%\tableofcontents

%\newpage

%\setcounter{section}{-1}

\section{BCS波函数}
证明BCS平均场基态的波函数可以写成如下形式:
$$ \ket{\Psi}=\prod_k\left(u_k+v_kc_{k\uparrow}^\dagger c_{-k\downarrow}^\dagger\right)\ket{0} $$
\paragraph{Solution:}

\section{超导体的格林函数}
\subparagraph{a)} 计算超导BCS平均场基态的格林函数$ G(\vec k,\omega) $,以及相应的电子谱函数$ A(\vec k,\omega) $。(可以比较一下计算得到的谱函数和ARPES实验的结果)
\subparagraph{b)} 计算电子的局域态密度$ \rho(\omega)=\dfrac{1}{V}\sum_k A(\vec k,\omega) $.(可以比较一下STM实验的结果)
\paragraph{Solution:}
\subparagraph{a)} 
\subparagraph{b)}


\end{document}