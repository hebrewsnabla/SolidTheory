%\documentclass[UTF8]{ctexart} % use larger type; default would be 10pt
\documentclass[a4paper]{article}
\usepackage{xeCJK}
%\usepackage{ctex}
%\usepackage{luatexja-fontspec}
%\setmainjfont{FandolSong}
%\usepackage[utf8]{inputenc} % set input encoding (not needed with XeLaTeX)

%%% Examples of Article customizations
% These packages are optional, depending whether you want the features they provide.
% See the LaTeX Companion or other references for full information.

%%% PAGE DIMENSIONS
\usepackage{geometry} % to change the page dimensions
\geometry{a4paper} % or letterpaper (US) or a5paper or....
\geometry{margin=1.2in} % for example, change the margins to 2 inches all round
% \geometry{landscape} % set up the page for landscape
%   read geometry.pdf for detailed page layout information

\usepackage{graphicx} % support the \includegraphics command and options

% \usepackage[parfill]{parskip} % Activate to begin paragraphs with an empty line rather than an indent

%%% PACKAGES
\usepackage{booktabs} % for much better looking tables
\usepackage{array} % for better arrays (eg matrices) in maths
\usepackage{paralist} % very flexible & customisable lists (eg. enumerate/itemize, etc.)
\usepackage{verbatim} % adds environment for commenting out blocks of text & for better verbatim
\usepackage{subfig} % make it possible to include more than one captioned figure/table in a single float
% These packages are all incorporated in the memoir class to one degree or another...

%%% HEADERS & FOOTERS
\usepackage{fancyhdr} % This should be set AFTER setting up the page geometry
\pagestyle{fancy} % options: empty , plain , fancy
\renewcommand{\headrulewidth}{0pt} % customise the layout...
\lhead{}\chead{}\rhead{}
\lfoot{}\cfoot{\thepage}\rfoot{}

%%% SECTION TITLE APPEARANCE
\usepackage{sectsty}
%\allsectionsfont{\sffamily\mdseries\upshape} % (See the fntguide.pdf for font help)
% (This matches ConTeXt defaults)

%%% ToC (table of contents) APPEARANCE
\usepackage[nottoc,notlof,notlot]{tocbibind} % Put the bibliography in the ToC
\usepackage[titles,subfigure]{tocloft} % Alter the style of the Table of Contents
%\renewcommand{\cftsecfont}{\rmfamily\mdseries\upshape}
%\renewcommand{\cftsecpagefont}{\rmfamily\mdseries\upshape} % No bold!

%%% END Article customizations

%%% The "real" document content comes below...

\setlength{\parindent}{0pt}
\usepackage{physics}
\usepackage{amsmath}
%\usepackage{symbols}
\usepackage{amsfonts}
\usepackage{bm}
%\usepackage{eucal}
\usepackage{mathrsfs}
\usepackage{amssymb}
\usepackage{float}
\usepackage{multicol}
\usepackage{abstract}
\usepackage{empheq}
\usepackage{extarrows}
\usepackage{textcomp}
\usepackage{mhchem}
\usepackage{braket}
\usepackage{siunitx}
\usepackage[utf8]{inputenc}
\usepackage{tikz-feynman}
\usepackage{feynmp}
\usepackage{fontspec}
%\sffamily
%\setmainfont{CMU Serif}
%\setsansfont{CMU Sans Serif}
%\setmonofont{CMU Typewriter Text}




\DeclareMathOperator{\p}{\prime}
\DeclareMathOperator{\ti}{\times}

\DeclareMathOperator{\e}{\mathrm{e}}
\DeclareMathOperator{\I}{\mathrm{i}}
\DeclareMathOperator{\Arg}{\mathrm{Arg}}
\newcommand{\NA}{N_\mathrm{A}}
\newcommand{\kB}{k_\mathrm{B}}

\DeclareMathOperator{\ra}{\rightarrow}
\DeclareMathOperator{\llra}{\longleftrightarrow}
\DeclareMathOperator{\lra}{\longrightarrow}
\DeclareMathOperator{\dlra}{\;\Leftrightarrow\;}
\DeclareMathOperator{\dra}{\;\Rightarrow\;}

%%%%%%%%%%%% QUANTUM MECHANICS %%%%%%%%%%%%%%%%%%%%%%%%
\newcommand{\bkk}[1]{\Braket{#1|#1}}
\newcommand{\bk}[2]{\Braket{#1|#2}}
\newcommand{\bkev}[2]{\Braket{#2|#1|#2}}

\DeclareMathOperator{\na}{\bm{\nabla}}
\DeclareMathOperator{\nna}{\nabla^2}
\DeclareMathOperator{\drrr}{\dd[3]\vb{r}}

\DeclareMathOperator{\psis}{\psi^\ast}
\DeclareMathOperator{\Psis}{\Psi^\ast}
\DeclareMathOperator{\hi}{\hat{\vb{i}}}
\DeclareMathOperator{\hj}{\hat{\vb{j}}}
\DeclareMathOperator{\hk}{\hat{\vb{k}}}
\DeclareMathOperator{\hr}{\hat{\vb{r}}}
\DeclareMathOperator{\hT}{\hat{\vb{T}}}
\DeclareMathOperator{\hH}{\hat{H}}

\DeclareMathOperator{\hL}{\hat{\vb{L}}}
\DeclareMathOperator{\hp}{\hat{\vb{p}}}
\DeclareMathOperator{\hx}{\hat{\vb{x}}}
\DeclareMathOperator{\ha}{\hat{\vb{a}}}
\DeclareMathOperator{\hs}{\hat{\vb{s}}}
\DeclareMathOperator{\hS}{\hat{\vb{S}}}
\DeclareMathOperator{\hSigma}{\hat{\bm\Sigma}}
\DeclareMathOperator{\hJ}{\hat{\vb{J}}}

\DeclareMathOperator{\Tdv}{-\dfrac{\hbar^2}{2m}\dv[2]{x}}
\DeclareMathOperator{\Tna}{-\dfrac{\hbar^2}{2m}\nabla^2}

%\DeclareMathOperator{\s}{\sum_{n=1}^{\infty}}
\DeclareMathOperator{\intinf}{\int_0^\infty}
\DeclareMathOperator{\intdinf}{\int_{-\infty}^\infty}
%\DeclareMathOperator{\suminf}{\sum_{n=0}^\infty}
\DeclareMathOperator{\sumnzinf}{\sum_{n=0}^\infty}
\DeclareMathOperator{\sumnoinf}{\sum_{n=1}^\infty}
\DeclareMathOperator{\sumndinf}{\sum_{n=-\infty}^\infty}
\DeclareMathOperator{\sumizinf}{\sum_{i=0}^\infty}

%%%%%%%%%%%%%%%%% PARTICLE PHYSICS %%%%%%%%%%%%%%%%
\DeclareMathOperator{\hh}{\hat{h}}               % helicity
\DeclareMathOperator{\hP}{\hat{\vb{P}}}          % Parity
\DeclareMathOperator{\hU}{\hat{U}}
\DeclareMathOperator{\hG}{\hat{G}}

\DeclareMathOperator{\GeV}{\si{GeV}}
\DeclareMathOperator{\LI}{\mathscr{L}.I.}
%\DeclareMathOperator{\g5}{\gamma^5}
\DeclareMathOperator{\gmuu}{\gamma^\mu}
\DeclareMathOperator{\gmud}{\gamma_\mu}
\DeclareMathOperator{\gnuu}{\gamma^\nu}
\DeclareMathOperator{\gnud}{\gamma_\nu}

\renewcommand{\u}{\mathrm{u}}
\renewcommand{\d}{\mathrm{d}}
\DeclareMathOperator{\s}{\mathrm{s}}

\DeclareMathOperator{\q}{\mathrm{q}}
\DeclareMathOperator{\bq}{\bar{\mathrm{q}}}

\DeclareMathOperator{\g}{\mathrm{g}}
\DeclareMathOperator{\W}{\mathrm{W}}
\DeclareMathOperator{\Z}{\mathrm{Z}}

%%% Feynman Diagram
\newcommand{\pa}{particle}
\newcommand{\mo}{momentum}
\newcommand{\el}{edge label}

%%% MQC
\DeclareMathOperator{\sH}{\mathscr{H}}
\DeclareMathOperator{\sA}{\mathscr{A}}
\newcommand{\iden}{{\large \bm{1}}}
\newcommand{\qed}{$ \Square $}
\newcommand{\tPhi}{\tilde{\Phi} }
\newcommand{\hsP}{\hat{\mathscr{P}}}
\newcommand{\hsS}{\hat{\mathscr{S}}}
\DeclareMathOperator{\core}{\mathrm{core}}
\DeclareMathOperator{\corr}{\mathrm{corr}}
\DeclareMathOperator{\ext}{\mathrm{ext}}

%ASSP
\DeclareMathOperator{\tb}{\text{TB}}

\newcommand{\dis}{\displaystyle}
\numberwithin{equation}{section}

\title{固体理论, Homework 05}
\author{王石嵘 20110220098}
\date{\today} % Activate to display a given date or no date (if empty),
         % otherwise the current date is printed 

\begin{document}
% \boldmath
\maketitle

%\tableofcontents

%\newpage

%\setcounter{section}{-1}


\section{
考虑一个包含四次非简谐项的一维振子链:}
\begin{equation}\label{key}
	 H = \sum_i\dfrac{P_i^2}{2M} + \sum_i \left[\dfrac{1}{2}M\omega_0^2(X_i-X_{i+1})^2 + \alpha(X_i-X_{i+1})^4 \right]
\end{equation}
将上述经典哈密顿量用正则量子化写成声子的二次量子化形式。(提示:利用哈密顿量中的二次项进行正则量子化,用和课程中一样的方法引入声子产生/湮灭算符,再将四次项表示成产生湮灭算符的形式。)
\paragraph{Solution:}
Let
\begin{equation}\label{key}
	\left\{ 
	\mqty{P_i &= \dfrac{1}{\sqrt{N}} \sum_k \e^{\I k n a} P_k\\
	X_i &= \dfrac{1}{\sqrt{N}} \sum_k \e^{\I k n a} X_k
} \right.
\end{equation}
thus, follow the textbook
\begin{align}
	\sum_i\dfrac{P_i^2}{2M} &= \dfrac{1}{2M}\sum_k P_{-k} P_k \\
	\sum_i \dfrac{1}{2}M\omega_0^2(X_i-X_{i+1})^2 &= \sum_k \dfrac{M\omega_k^2}{2} X_{-k}X_k
\end{align}
where
\begin{equation}\label{key}
	\omega_k^2 = 2\omega_0^2 (1 - \cos ka)
\end{equation}
the rest term
\begin{align}
	(X_i-X_{i+1})^4 &= (X_i^2 + X_{i+1}^2 - X_iX_{i+1} - X_{i+1}X_i)^2 \notag\\
	%&= \qty[\sum_k X_{-k}X_k (2 - \e^{\I ka} - \e^{-\I ka})]^2 \notag\\
	%&= \sum_{k,k'} X_{-k}X_k X_{-k'}X_{k'} (4 - 2\e^{\I ka} - 2\e^{-\I ka} - 2\e^{\I k'a} - 2\e^{-\I k'a} - \e^{\I (k+k')a} - \e^{-\I (k+k')a} - \e^{\I (k-k')a} - \e^{-\I (k-k')a} )
	&= X_{-k}X_k X_{-k}X_k \omega_k^4/\omega_0^4
\end{align}
Let
\begin{equation}\label{key}
	\tilde{P}_k = \dfrac{1}{\sqrt{2M\omega_k}} P_k \qquad \tilde{Q}_k = \sqrt{\dfrac{M\omega_k}{2}}X_k +  \dfrac{\sqrt{\alpha\omega_k^3}}{\omega_0^2}X_k X_k
\end{equation}

\begin{equation}\label{key}
	b_k = \tilde{Q}_k + \I\tilde{P}_k \qquad b_k^\dagger = \tilde{Q}_k - \I\tilde{P}_k 
\end{equation}
thus
\begin{align}\label{key}
	H &= \sum_k \omega_k (\tilde{P}_{-k}\tilde{P}_k + \tilde{Q}_{-k}\tilde{Q}_k)\\
	&= \sum_k \omega_k (b_k^\dagger b_k + \dfrac{1}{2})
\end{align}


\section{考虑课程中讨论的电子-晶格相互作用}
\begin{equation}\label{key}
	H_{ei}=\sum_{ij}V_{ei}\left(\vec r_j - \vec R_i\right)
\end{equation}
课上我们将$ V_{ei} $展开到$ \vec Q_i $的线性项得到了电声子相互作用。将$  V_{ei} $展开到下一阶( $ \vec Q_i $的平方项),计算下一阶的电声子相互作用。
\paragraph{Solution:}
\begin{align}\label{key}
	H_{ei} &= \sum_{ij} V_{ei}(\vec r_j - \vec R_i^0) + \sum_{ij} \pdv{V_{ei}}{R_i^0}\cdot \vec Q_i + \sum_{i,i'}\sum_j \vec Q_{i'} \pdv[2]{V_{ei}}{R_i^0}{R_{i'}^0} \vec Q_i \notag\\
	&= \sum_{ij} V_{ei}(\vec r_j - \vec R_i^0) + \sum_{ij} \nabla_j V_{ei}\cdot \vec Q_i + \sum_{i,i'}\sum_j \vec Q_{i'} [\nabla_j\otimes\nabla_j] V_{ei} \vec Q_i 
\end{align}
Here by $ [\nabla_j\otimes\nabla_j] V_{ei} $, we refer to a Hessian matrix of $ V_{ei}  $ with respect to $ \vec r_j $.\\
Do Fourier transformation
\begin{align}
	V_{ei} &= \dfrac{1}{\sqrt{N}} \sum_k V_{ei}(\vec k) \e^{\I \vec k \cdot (\vec r_j - \vec R_i^0)} \\
	\nabla_j V_{ei} &= \dfrac{1}{\sqrt{N}} \sum_k V_{ei}(\vec k) \e^{\I \vec k \cdot (\vec r_j - \vec R_i^0)} \I\vec k \\
	[\nabla_j\otimes\nabla_j] V_{ei} &= \dfrac{1}{\sqrt{N}} \sum_k V_{ei}(\vec k) \e^{\I \vec k \cdot (\vec r_j - \vec R_i^0)} [-\vec k\otimes\vec k]
\end{align}
thus
\begin{align}
	H_{e-ph}^1 &= \sum_{q,L,\lambda} M_{qL\lambda}(b_q +b_{-q}^\dagger) \rho_{q+L}
\end{align}
\begin{align}
	H_{e-ph}^2 &= \sum_j \sum_{i,k,\lambda} \dfrac{1}{\sqrt{2MN\omega_{k\lambda}}} (b_{k\lambda}\lambda_k \e^{\I \vec k \cdot \vec R_i^0} + b_{k\lambda}^\dagger\lambda_k^* \e^{-\I \vec k \cdot \vec R_i^0}) 
	\dfrac{1}{\sqrt{N}} \sum_{k''} V_{ei}(\vec k'') \e^{\I \vec k'' \cdot (\vec r_j - \vec R_i^0)} [-\vec k''\otimes\vec k''] \notag\\
	\quad &\sum_{i',k',\lambda'} \dfrac{1}{\sqrt{2MN\omega_{k\lambda}}} (b_{k'\lambda'}\lambda_{k'} \e^{\I \vec k' \cdot \vec R_{i'}^0} + b_{k'\lambda'}^\dagger\lambda_{k'}^* \e^{-\I \vec k' \cdot \vec R_{i'}^0}) 
\end{align}

\end{document}