%\documentclass[UTF8]{ctexart} % use larger type; default would be 10pt
\documentclass[a4paper]{article}
\usepackage{hw}

%ASSP
\DeclareMathOperator{\tb}{\text{TB}}

\newcommand{\dis}{\displaystyle}
\numberwithin{equation}{section}

\title{固体理论, Homework 08}
\author{王石嵘 20110220098}
\date{\today} % Activate to display a given date or no date (if empty),
         % otherwise the current date is printed 

\begin{document}
% \boldmath
\maketitle

%\tableofcontents

%\newpage

%\setcounter{section}{-1}

\section{教材13.1}
Consider a linear chain with a single defect placed at the mth site. Let the site
amplitude be $ c_n $ and $ W $ the strength of the defect in a linear chain described by the
following evolution equation:
\begin{equation}\label{key}
E c_n = \epsilon c_n +V (c_{n+1} + c_{n−1}) +W \delta_{nm}c_m. 
\end{equation}
Use the method of defects to show that a single bound state forms outside the
continuous band for $ W \neq 0 $. To implement this method, use the following procedure.\\
(1) Multiply the eigenvalue equation by $ \e^{ikn} $ and sum over $ n $.\\ (2) Then multiply the
resultant equation by $ \e^{−ikm} $ and integrate to obtain as a condition for the location of
the bound state
\begin{equation}\label{key}
 1 = W\ev{G(E)} 
\end{equation}
where
\begin{equation}\label{key}
\ev{G(E)} = \dfrac{1}{2\pi} \int_{-\pi}^\pi \dfrac{\dd k}{E - \epsilon(k)} 
\end{equation}

with $ \epsilon(k) = \epsilon + 2V \cos k $. Show that if $ W > 0 $, the bound state lies above the band,
that is, at an energy $ E > \epsilon + 2V $ , whereas for $ W < 0 $, the bound state lies below the
band. Repeat the same calculation for $ d = 2 $.
\paragraph{Solution:}
\begin{equation}\label{key}
\sum_n(E - \epsilon )c_n\e^{ikn} =  \sum_n V (c_{n+1} + c_{n−1}) \e^{ikn} + \sum_n W \delta_{nm}c_m \e^{ikn} 
\end{equation}
\begin{equation}\label{key}
\sum_n(E - \epsilon )c_n\e^{ikn} =  \sum_n V (c_{n+1} + c_{n−1}) \e^{ikn} + W c_m \e^{ikm} 
\end{equation}
\begin{equation}\label{key}
 \sum_n(E - \epsilon )c_n\e^{ikn} \e^{-ikm} =   \sum_n V (c_{n+1} + c_{n−1}) \e^{ikn}\e^{-ikm} +  W c_m \e^{ikm} \e^{-ikm}
\end{equation}
\begin{equation}\label{key}
 \sum_n(E - \epsilon )c_n\e^{ik(n-m)}  =  \sum_n V (c_{n+1} + c_{n−1}) \e^{ik(n-m)} +  W c_m 
\end{equation}
\begin{equation}\label{key}
\int_{-\pi}^\pi \dd k \dfrac{1}{W} = \int_{-\pi}^\pi \dd k \dfrac{1}{(E-\epsilon) - (V\e^{ik} + V\e^{-ik})}
\end{equation}
\begin{equation}\label{key}
\dfrac{1}{W} = \dfrac{1}{2\pi} \int_{-\pi}^\pi \dd k \dfrac{1}{E- \epsilon - 2V\cos k }
\end{equation}
thus
\begin{equation}\label{key}
\ev{G(E)} =  \dfrac{1}{2\pi} \int_{-\pi}^\pi \dd k \dfrac{1}{E- \epsilon(k) }
\end{equation}
where $ \epsilon(k) = \epsilon + 2V\cos k $
\newpage
\section{教材13.5}
Assume that in the Ohmic regime, the general form for the $ \beta $-function is
\begin{equation}\label{key}
 \beta = d - 2 - \dfrac{A_d}{g} 
\end{equation}
Integrate this quantity in the interval $ [L_0, L] $ and obtain the explicit length dependence
for the conductance for $ d = 1 $ and $ d = 3 $, respectively. Show that $ s = 1 $, provided
that $ \epsilon = d - 2 \ll 1 $.
\paragraph{Solution:}
Since
\begin{equation}\label{key}
\beta = \dv{\ln g(L)}{\ln L}
\end{equation}
we have
\begin{equation}\label{key}
 \beta \dd\ln L =  \dd \ln g
\end{equation}
\begin{equation}\label{key}
 \qty(d - 2 - \dfrac{A_d}{g}) \dfrac{1}{L}\dd L = \dfrac{1}{g}\dd  g
\end{equation}
\begin{equation}\label{key}
\qty(d - 2 - \dfrac{A_d}{g}) \dfrac{g}{L} = \dv{g}{L}
\end{equation}
%\begin{equation}\label{key}
%\int_{L_0}^L \qty((d - 2)g - {A_d}) \dfrac{1}{L}\dd L = \int_{L_0}^L \dd  g
%\end{equation}
%\begin{equation}\label{key}
%\int_{L_0}^L (d - 2)\dfrac{g}{L}\dd L - A_d \ln (L/L_0) = g(L) - g(L_0)
%\end{equation}
solve it, we get
\begin{equation}\label{key}
g(L) = \dfrac{A_g}{d - 2} + c L^{d-2}
\end{equation}
thus
\begin{equation}\label{key}
\dfrac{g(L) - g(0)}{g(L_0) - g(0)} = \qty(\dfrac{L}{L_0})^{d-2}
\end{equation}
where $ g(0) = \dfrac{A_g}{d-2} = g_c $.\\
Here we get 
\begin{equation}\label{key}
 \nu = \dfrac{1}{d-2} 
\end{equation}
thus
\begin{equation}\label{key}
s = (d-2)\nu = 1
\end{equation}



% \int_{L_0}^L
\end{document}